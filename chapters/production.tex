\chapter{Uvedení do provozu}

Po ukončení testování byla aplikace uvedena do provozu na přelomu května a června 2023 a bude využita
pro nadcházející ročník akce SummerJob. Tato kapitola popisuje
technické řešení nasazení aplikace na produkční server a popis spuštění.

\section{Zvolené řešení}

V současné době je původní webová komponenta s databází provozována na domácím serveru,
který je připojen k internetu pomocí pevné IP adresy. Tento server je však již zastaralý a
nedostačující pro provoz nové aplikace. Proto bylo rozhodnuto o spuštění nové aplikace na
externím serveru. Pro tento účel byl vybrán server poskytovatele Hukot.net\footnote{Hukot.net, \url{https://www.hukot.net/}}
v konfiguraci \textit{VPS-L02G} s operačním systémem Ubuntu Server 22.04 LTS. Tento server
poskytuje 2 GB operační paměti, 20 GB úložiště a jedno vlákno procesoru s možností přechodu
na vyšší konfiguraci v případě potřeby. Server je připojen k internetu pomocí pevné IP adresy.

Aplikace je nasazena na serveru pomocí Docker kontejnerů. To umožňuje snadné nasazení aplikace
na jiném serveru v případě potřeby. V případě selhání některého z kontejnerů dojde k automatickému
restartu kontejneru, což zajišťuje dostupnost i v případě neočekávané chyby.

Pro propojení aplikace s doménou \url{www.summerjob.eu} byl využit DNS záznam typu A pro subdoménu
\texttt{jobplanner.summerjob.eu}, který směřuje na IP adresu serveru. Na serveru byl následně
nastaven kontejner SWAG\footnote{Secure Web Application Gateway, \url{https://docs.linuxserver.io/general/swag}}, který obsahuje nástroje pro bezpečný provoz webových aplikací.
Hlavní komponentou kontejneru je webový server Nginx v režimu reverzní proxy, který předává veškeré požadavky na subdoménu
\href{https://jobplanner.summerjob.eu}{jobplanner.summerjob.eu} na kontejner s aplikací. Reverzní proxy je spuštěna v kontejneru
společně s nástrojem \texttt{certbot}, který zajišťuje automatické obnovování SSL certifikátů pro subdoménu
pomocí certifikační autority Let's Encrypt. Veškerá komunikace s aplikací je tedy šifrována, požadavky na
nešifrované spojení jsou automaticky přesměrovány. 

\section{Proces nasazení}

Pro nasazení aplikace na produkční server byl využit nástroj \texttt{docker-compose}, který
umožňuje spouštět a spravovat skupiny kontejnerů pomocí jednotného konfiguračního souboru.
Konfigurace je zapsána v souboru \texttt{docker-compose.yaml}, který je součástí repozitáře aplikace.
Tento soubor obsahuje konfiguraci pro kontejnery s aplikací, databází a webovým serverem, kterou 
je před spuštěním nutné upravit podle návodu, který je součástí repozitáře.

Jedná se zejména o nastavení databáze a přístupových údajů pro e-mailový účet, který je využíván
pro odesílání přihlašovacích e-mailů. Celý proces včetně příkazů a popisu proměnných prostředí je popsán
v repozitáři aplikace v souboru \texttt{README.md}.

\section{Inicializace databáze}

Po spuštění kontejnerů je nutné provést inicializaci databáze
a vytvoření uživatelského účtu s administrátorskými právy. K tomuto účelu byl vytvořen inicializační kontejner, který
obsahuje skripty potřebné pro inicializaci a správu databáze. Kontejner obsahuje skripty pro vytvoření potřebných tabulek v databázi,
přidání prvního administrátorského účtu, nainstalovaný nástroj Prisma CLI a další.

Jedná se o servisní kontejner určený pro jednorázové interaktivní spuštění. Kontejner umožňuje také přístup k databázi
pomocí nástroje Prisma Studio, který umožňuje správu dat v databázi pomocí webového rozhraní. 

\section{Zálohování}

K zálohování databáze je možné využít standardní nástroje pro zálohování databáze PostgreSQL.
Vhodným řešením je například aplikace pgAdmin\footnote{\url{https://www.pgadmin.org/}}, kterou
je možné spustit v kontejneru propojeném se sítí databáze.
Ve výchozím nastavení není databáze automaticky zálohována.

\section{Škálování}

 Aplikaci bude využívat
pravidelně zejména organizační tým akce, který se skládá přibližně z 25 členů. Z analýzy počtu dobrovolníků v předchozích ročnících akce SummerJob vyplývá, že
se ročníku účastní průměrně 150 dobrovolníků. Tito dobrovolníci budou aplikaci využívat jednorázově na nastavení údajů a kontrolu plánu.
Vzhledem k počtu uživatelů a jejich aktivitám není nutné řešit automatické škálování aplikace. Pokud by však bylo nezbytné
zvýšit výkon aplikace, je možné navýšit hardwarové prostředky serveru přechodem na vyšší konfiguraci.