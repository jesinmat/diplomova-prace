\begin{introduction}
	SummerJob je letní dobrovolnická brigáda, která probíhá
ve vesnicích a městečkách v českém pohraničí. Akce se každý rok účastní přibližně 150 brigádníků, kteří se podílejí na plnění požadovaných prací.
Obyvatelé vesnic a městeček mohou zdarma kontaktovat správce této akce a požádat o pomoc s nějakým úkolem.
Brigádníci spolu s týmem organizátorů poté na daném místě vykonávají požadované práce v období jednoho týdne.

Tyto práce jsou rozmístěny v různých oblastech -- jedná se typicky o sousedící vesnice či městečka, mezi kterými je nutné se přepravovat autem.
Aby bylo možné efektivně plánovat brigádníky na jednotlivé práce, je nutné vytvořit systém, který evidenci a plánování usnadní.

V současnosti pro tyto účely používají organizátoři akce SummerJob vlastní aplikaci, která je dostupná pouze na počítači. Tato aplikace je však podle
názoru organizátorů nestabilní a nevyhovuje jejich požadavkům. Pro zjednodušení celého procesu potřebného ke správnému fungování akce je nutné vytvořit novou webovou aplikaci,
která bude plnit stejné funkce jako
současná aplikace, ale bude řešit její nedostatky. Nová aplikace bude dostupná na mobilních zařízeních, bude mít jednodušší a přehlednější rozhraní, bude
podporovat práci více uživatelů současně a celkově bude lépe přizpůsobena potřebám organizátorů akce.
\end{introduction}