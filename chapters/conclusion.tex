\begin{conclusion}
	Cílem této práce bylo navrhnout a implementovat systém pro správu dobrovolnické brigády SummerJob.
	Vytvořená aplikace umožňuje správu plánů, brigádníků, oblastí, jednotlivých prací, dopravních prostředků potřebných pro výkon prací a dalších souvisejících částí.
	Obsahuje také automatický plánovací systém, který umožňuje přiřazovat brigádníky na jednotlivé práce podle zadaných parametrů.

	Po analýze stávajícího řešení bylo zjištěno, že je současná aplikace zastaralá a nevyhovuje požadavkům organizátorů.
	Na základě analýzy požadavků a technologií byla navržena nová webová aplikace, která je postavena na moderních technologiích a umožňuje snadnou rozšiřitelnost.
	Pro implementaci byl zvolen framework Next.js, který umožňuje vytvořit webovou aplikaci v technologii React a serverovou část včetně API.
	Automatický plánovací systém byl implementován jako samostatná komponenta, aby bylo možné jej snadno rozšířit či nahradit.
	Webová aplikace také umožňuje přihlášení brigádníkům pro zobrazení jejich denního plánu.

	Aplikace byla po dokončení implementace a testování nasazena na produkční server a bude využívána při organizaci následujících akcí SummerJob.

	Cíle této práce byly splněny, vytvořená aplikace splňuje požadavky organizátorů a umožňuje snadnou správu brigádníků a plánování prací.

\end{conclusion}
