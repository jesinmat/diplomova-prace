\chapter{Analýza a návrh}

Tato kapitola popisuje proces analýzy požadavků a návrhu systému.
V první části jsou popsány metody použité při analýze požadavků. V druhé části jsou popsány návrhy systému, které byly vytvořeny na základě analýzy požadavků.
Analýza je nezávislá na technologiích použitých při následné implementaci.

\section{Analýza požadavků}

Analýza požadavků je proces, který má za cíl získat informace o požadavcích na systém. Na základě opakovaných konzultací s vedoucím práce, Ing. Markem Jílkem, a 
Bc. Lucií Annou Procházkovou, organizátory akce SummerJob, byly získány podrobné informace o procesu organizace akcí a požadované funkcionalitě nové aplikace.
Tyto konzultace byly prováděny i dále během vývoje systému, aby bylo zajištěno, že systém bude plnit požadavky organizátorů akce. Pomocí této agilní metody
bylo možné průběžně reagovat na případné změny a seznamovat budoucí uživatele s aktuálním stavem systému.

V této práci bude kromě obecných pojmů pro jednotlivé součásti systému použita také existující terminologie z akce SummerJob: \textit{pracant} je označení pro brigádníka, \textit{job} je označení pro práci, kterou brigádník vykonává.

\subsection{Funkční požadavky}

Funkční požadavky popisují funkcionalitu požadované aplikace, specifikují možnosti, vlastnosti a operace, které musí být možné provádět.
Jsou základem při návrhu a implementaci systému.

\begin{itemize}
    \item \textbf{FP1. Přihlášení a registrace:} uživatelé systému musí být schopni se přihlásit do systému, pokud jsou registrováni v právě probíhajícím ročníku. Registraci není třeba implementovat, probíhá externě a schvalují ji organizátoři akce.
    \item \textbf{FP2. Správa oprávnění a uživatelských účtů:} uživatelé systému mají různá oprávnění, která určují, co všechno mohou v systému provádět. Oprávnění jsou rozdělena podle přístupu ke správně jednotlivých položek (pracanti, joby, auta, plánování). Administrátor má možnost nastavit oprávnění jednotlivým uživatelům, případně uživateli zcela zakázat přístup do systému.
    \item \textbf{FP3. Správa pracantů:} organizátoři akce musí být schopni spravovat seznam pracantů. Musí být možné přidávat, mazat a upravovat informace o pracantech. Tyto informace zahrnují jméno a příjmení, e-mail, telefonní číslo, alergie, dostupnost pracanta v jednotlivých dnech akce, informaci, zda a kdy se chce pracant účastnit adorací. Organizátoři sami během akce pracují a patří tedy mezi pracanty.
    \item \textbf{FP4. Správa jobů:} organizátoři akce musí být schopni spravovat seznam jobů. Musí být možné přidávat, mazat a upravovat informace o jobech. Tyto informace zahrnují název jobu, popis jobu, lokalitu, kontaktní osobu, alergeny na místě, celkový počet pracantů, které je možné přiřadit k jobu, požadovaný počet silných pracantů a počet dní nutných na splnění práce. Jednotlivé joby musí být možné označit jako splněné, případně nežádoucí. Joby musí být také možné označit jako prioritní. Prioritní joby musí být v seznamu jobů odlišeny od ostatních jobů a slouží k zvýšení viditelnosti jobů, které jsou pro organizátory důležité. V jobech musí být možné vyhledávat.
    \item \textbf{FP5. Správa aut:} organizátoři mohou spravovat dostupná auta. U aut se eviduje počet míst, vlastník (pracant) a najeté kilometry. Na základě najetých kilometrů během akce se po skončení akce počítá kompenzace pro řidiče. Systém by měl umožnit evidenci, zda byla kompenzace vyplacena.
    \item \textbf{FP6. Správa plánů:} zodpovědné osoby mohou tvořit plány na jednotlivé dny, přidávat joby do plánů, přidávat pracanty do jobů, plánovat dopravu pro jednotlivé joby. Pracanti z různých jobů mohou sdílet dopravu. Systém musí podporovat současnou práci více uživatelů během tvorby plánu. Plány musí být možné zobrazit v tisknutelném formátu.
    \item \textbf{FP7. Automatický plánovač:} systém musí být schopný na žádost oprávněného uživatele naplánovat rozvrh pracantů na zadané joby. Po přidání jobů do plánu a vyžádání vygenerování plánu plánovač přiřadí pracanty na joby a naplánuje dopravu. Přitom respektuje dostupnost pracantů v daný den, alergie, požadovaný počet silných pracantů na jobu a požadavky na adoraci.
    \item \textbf{FP8. Správa ročníků:} systém musí být schopný spravovat více ročníků akce SummerJob. Administrátor systému má možnost přepínat mezi jednotlivými ročníky. Každý ročník má název a časové rozpětí.
    \item \textbf{FP9. Správa oblastí:} systém musí být schopný spravovat oblasti, do kterých jsou joby rozděleny. U oblasti se eviduje nutnost dopravy autem a možnost adorace. Oblasti jsou vztaženy k aktuálnímu ročníku.
    \item \textbf{FP10. Záznam systémových změn:} systém musí být schopný zaznamenávat změny, které uživatelé provádějí. Záznamy musí obsahovat časové razítko, uživatele, který změnu provedl a popis změny. Administrátor má možnost zobrazit záznamy změn.
    \item \textbf{FP11. API:} systém poskytuje API, které umožňuje přístup k datům systému. API musí umožňovat provádět operace na úrovni srovnatelné s webovou aplikací.
\end{itemize}

\subsection{Nefunkční požadavky}

Nefunkční požadavky se týkají výkonu, spolehlivosti, bezpečnosti, škálovatelnosti a dalších aspektů systému. Nepopisují funkce systému, ale zaměřují se na celkovou kvalitu, provozuschopnost a výkonnost.

\begin{itemize}
    \item \textbf{NP1. Bezpečnost:} systém bude využívat přihlášení a ověření práv uživatelů pro přístup k datům. Uživatel bez potřebného oprávnění nesmí mít přístup datům, pro které je dané oprávnění nezbytné. Výjimkou je přístup k datům potřebným pro vykonávání činností, na které má uživatel oprávnění, například přístup ke jménům pracantů při přidávání nového auta do systému za účelem určení majitele. Přístup k databázi nebude možný přímo z internetu, ale pouze přes webovou aplikaci nebo API. Spojení mezi uživatelem a aplikací musí být šifrované.
    \item \textbf{NP2. Udržovatelnost a rozšiřitelnost:} výsledná aplikace využívá rozšířené technologie a standardy, které umožňují další rozšíření. K dispozici je dokumentace popisující fungování aplikace a použité technologie. Vývojáři systému mají možnost snadno rozšiřovat aplikaci o nové funkce.
    \item \textbf{NP3. Nezávislost na platformě:} aplikace nesmí být vázána na konkrétní operační systém. Vývojáři systému mají možnost snadno přenést aplikaci na jinou platformu. Webová aplikace musí být responzivní a použitelná na současných mobilních zařízeních i počítačích.
    \item \textbf{NP4. Rychlost plánovače:} plánovač musí být schopen vygenerovat plán pro 150 pracantů a 50 jobů do 15 sekund.
\end{itemize}

\section{Návrh systému}

V této kapitole je popsán návrh systému na základě provedené analýzy. Návrh popisuje celkovou architekturu systému, webovou aplikaci a API, databázi a plánovací systém.

\subsection{Architektura systému}

\subsection{Webová aplikace a API}

\subsection{Databáze}

\subsection{Plánovací systém}