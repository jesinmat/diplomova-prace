\chapter{Implementace}

popsat vývoj webu, znova ukázat schéma db + ukázky Prismy, vysvětlit fungování Plánovače

Tato kapitola popisuje implementaci řešení. V první části je popsán vývoj klientské a serverové části webové aplikace včetně API, následující části se 
věnují implementaci Plánovače a kontejnerizaci.

\section{Vývoj webové aplikace}

Vývoj webové aplikace probíhal v iteracích, během kterých byly postupně implementovány jednotlivé funkce. Zástupci organizátorů akce SummerJob
měli k aplikaci přístup od začátku vývoje, což umožnilo průběžné testování a získávání zpětné vazby. Aplikace byla během vývoje dostupná na dočasné webové adrese.

Vybraný framework Next.js využívá rozšířenou verzi technologie React pro implementaci klientské části.
Pro zajištění správné funkcionality aplikace bylo nutné vytvořit několik vlastních komponent,
které jsou využívány v celé aplikaci. Jedná se zejména o tabulky se seznamy registrovaných brigádníků,
nabídek prací, evidencí aut a manuální úpravy plánu.
Klientská část webové aplikace je rozdělena do několika částí, které jsou popsány níže.

\subsection{Přihlašování}

Správa přihlašování je implementována pomocí knihovny NextAuth.js, která poskytuje rozhraní pro přihlašování pomocí různých poskytovatelů. Knihovna podporuje
přihlašování pomocí e-mailu, služeb Google, Facebook, GitHub, Twitter, Apple a dalších. Přihlašování pomocí uživatelského jména a hesla je autory knihovny 
považováno za nebezpečné a pro podporu je nutné implementovat vlastní ověření uživatelů. Na základě analýzy bylo proto zvoleno přihlašování pomocí e-mailu.

Přihlašování je první částí aplikace, která se zobrazí nepřihlášenému uživateli. Pokus o přístup na stránku, která vyžaduje přihlášení,
je zachycen na straně serveru a uživatel je přesměrován na přihlašovací stránku. Nedochází tedy k úniku citlivých dat nebo jiného interního obsahu.

Přihlašovací stránka obsahuje formulář s polem pro zadání e-mailové adresy. Po odeslání formuláře dojde k ověření existence uživatele v databázi.
Pokud je uživatel registrovaný v aktuálním ročníku akce a nejedná se o blokovaný účet, je uživateli odeslán e-mail s odkazem pro přihlášení a na stránce je 
zobrazena příslušná informace. Pokud uživatel neexistuje nebo je jeho účet blokovaný, je zobrazena chybová hláška.

Přihlášení pomocí odkazu v e-mailu je zabezpečeno pomocí jednorázového tokenu s omezenou časovou platností, který je generován při odeslání e-mailu.
Po přihlášení je token z databáze odstraněn.
Webovému prohlížeči je přiřazeno identifikační cookie s platností jeden měsíc, které je využíváno pro autentizaci uživatele. Během této doby se uživatel nemusí v
daném prohlížeči znovu přihlašovat. Platnost po dobu jednoho měsíce byla zvolena s ohledem na dobu trvání akce SummerJob -- jedná se o přípravu na ročník,
týden práce a čas pro vyřízení administrativy po skončení akce.

Cookie využívá atributy \texttt{HttpOnly} a \texttt{SameSite}, které zabraňují přístupu k hodnotě cookie ze skriptů a
pomáhají předcházet útokům typu CSRF. Dále je cookie chráněno proti úniku přes nezabezpečené připojení pomocí atributu \texttt{Secure}. 

Registrace dobrovolníků je řešena externě a organizátoři mají možnost importovat seznam vybraných dobrovolníků do databáze v záložce \textit{Pracanti}.

\subsection{Pracanti}

Záložka \textit{Pracanti} obsahuje seznam registrovaných brigádníků. Seznam je zobrazen v tabulce, která umožňuje vyhledávání a filtrování dat.
Jsou zde zobrazeni pouze brigádníci, kteří jsou registrovaní v aktuálním ročníku akce. U každého brigádníka je zobrazeno jméno, e-mail, telefonní číslo,
e-mailová adresa a speciální vlastnosti. Tyto vlastnosti v současnosti zahrnují informaci o tom, zda má brigádník k dispozici auto a zda je brigádník silný,
což je důležité pro plánování prací.

U každého pracanta jsou dále evidovány alergie a dny, kdy může pracovat. Tyto informace je možné zobrazit a upravit na samostatné stránce, která je dostupná
po kliknutí na ikonu úprav v příslušném řádku tabulky. Na této stránce je také možné zobrazit a upravit přihlašovací údaje brigádníka. Pro zobrazení a úpravu
pracantů je nutné mít příslušná práva, bez kterých není možné ani přistoupit na stránku.

Stránka umožňuje také přidávat nového brigádníka do databáze. Přidání nového brigádníka je řešeno pomocí formuláře, který obsahuje pole pro všechny evidované
hodnoty. V případě neplatného vstupu je uživatel upozorněn chybovou hláškou. Validace je řešena na straně klienta pomocí knihovny Zod, která umožňuje definovat
schéma dat a následně je použít pro validaci vstupu. Validace je také řešena na straně serveru stejným způsobem. Kromě individuálního přidávání je možné využít
funkci pro hromadný import, která umožňuje vložení seznamu brigádníků a jejich údajů ve formátu dat oddělených středníkem. Uživateli je před odesláním
k dispozici náhled importovaných dat.

Na žádost organizátorů byla do aplikace přidána možnost tisku seznamu brigádníků v jednoduché tabulce. Tato funkce je dostupná v záložce \textit{Pracanti} po kliknutí
na tlačítko \textit{Tisknout}.

\subsection{Joby}

V záložce \textit{Joby} jsou zobrazeny všechny práce, které je možné vykonat během akce. U každé práce je evidován název, popis, minimální
a maximální počet brigádníků, kteří mohou danou práci vykonávat současně, a počet silných brigádníků, kteří jsou pro práci potřeba. Dále jsou evidovány
informace o lokalitě, kde se práce bude vykonávat, včetně přesné adresy, kontakt na zadavatele, počet dnů potřebných pro vykonání práce, 
seznam dní, kdy je možné práci vykonávat, a alergeny v místě pracoviště. Evidence alergenů umožní automatickému plánovači i organizátorům přiřadit na práci brigádníky, kteří nemají
konfliktní alergie s alergeny na pracovišti. 

Do seznamu je možné přidávat nové práce a upravovat již existující, vyhledávat a filtrovat.
Jednotlivé joby je možné označit jako splněné nebo nežádoucí, což umožňuje organizátorům filtrovat práce, které již nejsou potřeba. Takové práce jsou v seznamu
zařazeny na konec stránky. Připnutí naopak umožňuje organizátorům označit práci jako důležitou a zobrazit ji na začátku seznamu. Označené joby jsou od standardních
barevně odlišeny.

\subsection{Auta}

Záložka \textit{Auta} obsahuje seznam aut, která jsou k dispozici pro přepravu brigádníků na pracoviště. U každého auta je evidován název, počet míst, majitel,
najeté kilometry a poznámka. Majitelem je vždy některý z registrovaných brigádníků. Na stránce je možné přidat, upravovat a odebírat auta.

Pro vyplacení kompenzace za použití auta je nutné evidovat počet najetých kilometrů. Na začátku je do systému zaznamenán stav odometru a po skončení akce
se na základě tohoto údaje a aktuálního stavu odometru vypočítá počet najetých kilometrů. Tato hodnota je následně použita pro výpočet kompenzace. Aplikace 
umožňuje evidovat výši kompenzace a údaj, zda došlo k proplacení.

\subsection{Plány}

Záložka \textit{Plány} obsahuje seznam plánů, které byly vytvořeny pro aktuální ročník akce. Na každý den akce vzniká nový plán, který zahrnuje
seznam brigádníků a jejich přiřazení na práce.

Plán je možné vytvořit ručně nebo automaticky. Ruční vytvoření plánu je řešeno pomocí formuláře, který umožňuje vybrat práce pro daný den a následně 
k nim přiřadit brigádníky. Organizátoři mohou naplánovat dopravu na pracoviště, systém umožňuje také sdílet dopravu s jinou prací. K vytvořenému záznamu
o práci je možné přidávat poznámky. Z přiřazených brigádníků je vybrána zodpovědná osoba, která zodpovídá za komunikaci s organizátory a zadavatelem a za správné vykonání práce.

Aplikace automaticky rozpoznává konflikty v plánu a upozorní organizátory pomocí varovného symbolu v řádku tabulky.
Konflikty mohou vzniknout v případě, že je na práci přiřazen nedostatek nebo přebytek brigádníků, 
chybí doprava, není přiřazena zodpovědná osoba, je přiřazen pracovník s konfliktní alergií, nebo je přiřazen pracovník, který projevil zájem o adoraci v daný den,
ale v dané lokalitě není možné požadavek splnit.

Stránka s plánem zobrazuje také statistiky pro daný den. Jsou zde zobrazeny informace o počtu brigádníků, kteří jsou přiřazeni na práci, počtu brigádníků, kteří
práci nemají, počet prací v plánu a rozsah počtu pracovníků, které lze daný den na vybrané práce přiřadit.

Je zde možnost využít automatické plánování, které je řešeno pomocí algoritmu, který je popsán v kapitole \ref{sec:planner}. Po výběru prací na daný den a spuštění
plánovače dojde k automatickému přiřazení brigádníků na práce tak, aby nebyl vytvořen žádný konflikt. Pokud je v plánu již nějaký brigádník přiřazen na práci, 
plánovač přiřazení zachová. Dojde také k naplánování dopravy a přiřazení zodpovědné osoby. Výjimkou v plánování je přiřazení brigádníků, kteří projevili zájem o adoraci,
na práci v lokalitě, kde není možné požadavek splnit -- plánovač může takové přiřazení provést, pokud není možné přiřadit brigádníka na práci ve vhodnější lokalitě.

Plánování podporuje práci více uživatelů současně. Během plánování dochází na pozadí k pravidelné aktualizaci dat, aby bylo zajištěno,
že uživatelé pracují s nejnovější verzí plánu. K aktualizaci dochází v intervalu 1 sekundy a k přenosu dat dojde pouze tehdy, pokud došlo od posledního požadavku ke změně,
čímž je minimalizován počet přenesených dat.

Aplikace umožňuje zobrazit plán v tisknutelné podobě. Jedná se o zjednodušenou černobílou verzi plánu v kompaktním zobrazení, které je vhodné pro tisk na 
papír velikosti A4. Dodatečné CSS styly zajišťují správné rozdělení plánu na papír tak, aby naplánovaná práce nepřesahovala na více stran současně.

\subsection{Administrace}

Záložka \textit{Administrace} slouží k nastavení aktuálního ročníku a oblastí, ve kterých se akce koná. Je zde také možné upravovat
práva uživatelů a prohlížet záznamy o změnách v systému, tzv. \textit{audit log}. Záznamy zahrnují veškeré akce, které mění data v databázi,
například vytvoření, úpravu nebo smazání záznamu. Záznamy jsou řazeny chronologicky a obsahují informace o uživateli, který změnu provedl,
o čase změny a o změněných datech. V záznamech je možné vyhledávat a filtrovat podle typu události.

\subsection{Můj plán}

Záložka \textit{Můj plán} zobrazuje seznam prací, na které je uživatel přiřazen. Brigádník může zobrazit detail práce pro daný den,
který obsahuje informace místě a popis práce, jména dalších brigádníků přiřazených na práci a způsob dopravy.

\subsection{Profil}

Záložka \textit{Profil} zobrazuje informace o uživateli. Uživatel může upravit své osobní údaje, označit dny, kdy může pracovat a adorovat,
a nastavit své případné alergie.
