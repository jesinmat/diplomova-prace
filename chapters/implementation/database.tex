\section{Databáze}

Přístup k databázi z aplikace a plánovače je implementován pomocí knihovny Prisma. Prisma je moderní, rychle se rozvíjející ORM framework pro práci s různými typy databází \cite{prisma}. 
Zakládá si na snadné integraci s jazykem TypeScript, díky kterému umožňuje pracovat s daty v databázích bezpečně. Podporuje také Javascript.

Jedná se o poměrně nový produkt -- první verze byla zveřejněna v roce 2019. 
Díky rozsáhlé komunitě a otevřenému zdrojovému kódu však dochází k neustálému zlepšování.
V současné době jeho popularita rychle roste a podle statistik webu \href{https://npmjs.com}{NPM.js} dosahuje přibližně 1 200 000 stažení týdně (na začátku roku 2022 to bylo 200 000) \cite{npm_prisma}.

Prisma umožňuje spravovat strukturu dat v databázi pomocí tzv. datového modelu (schématu), který je definován v souboru \texttt{schema.prisma}. 
Schéma je možné navrhnout manuálně, pokud databáze neexistuje, případně vygenerovat z existující databáze. Schéma se skládá z entit a jejich atributů a vztahů mezi nimi.
Výňatek ze souboru je k dispozici ve výpisu \ref{fig:prisma-model}.

\begin{listing}[h]
\begin{minted}{js}
datasource db {
    provider = "postgresql"
    url      = env("DATABASE_URL")
}

model Ride {
  id         String    @id @default(uuid())
  driver     Worker    @relation("Driver",
                        fields: [driverId],
                        references: [id],
                        onDelete: Cascade)
  driverId   String
  car        Car       @relation(fields: [carId], 
                        references: [id],
                        onDelete: Cascade)
  carId      String
  passengers Worker[]
  job        ActiveJob @relation(fields: [jobId], 
                        references: [id],
                        onDelete: Cascade)
  jobId      String
}
\end{minted}
\caption{Ukázka datového modelu Prisma}
\label{fig:prisma-model}
\end{listing}

Po definici schématu je možné vygenerovat příslušné typy a funkce pro jazyk TypeScript, které umožňují snadnou práci s databází.
Typy pro všechny dostupné operace jsou vytvořeny tak, aby odpovídaly datovému modelu a nebylo možné vytvořit nevalidní dotaz.
Během dotazování je možné využívat vazby mezi entitami a číst či měnit hodnoty atributů ve více tabulkách najednou.
Ukázka použití v aplikaci je na obrázku \ref{fig:prisma-query}.

\begin{listing}[h]
\begin{minted}{js}
const prisma = new PrismaClient();

async function getRidesOfDriver(driverId: string) {
    return await prisma.ride.findMany({
        where: {    
            driver: {
                id: driverId,
            },
        },
        include: {
            driver: true,
            car: true,
            passengers: true,
            job: true,
        },
    });
}
\end{minted}
\caption{Ukázka použití Prisma ORM}
\label{fig:prisma-query}
\end{listing}