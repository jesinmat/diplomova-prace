\section{Kontejnerizace}

Vytvořená aplikace je kontejnerizována pomocí nástroje Docker. Celkem se skládá ze čtyř kontejnerů pro provoz služeb a jednoho servisního kontejneru.
Repozitář obsahuje soubor \texttt{docker-compose.yaml}, který definuje jednotlivé služby a jejich konfiguraci a obsahuje také návod ke spuštění aplikace.
Vytvořené obrazy nejsou uloženy ve veřejném repozitáři, je nutné je před spuštěním aplikace sestavit na serveru.

\subsection{Kontejnerizace služeb}

Pro ostrý provoz webové aplikace byl vytvořen kontejner vycházející z obrazu \texttt{node:18-slim}. Tento obraz obsahuje 
Node.js ve verzi 18 LTS, která byla vybrána z důvodu aktivního vývoje a dlouhodobé podpory, a je založen na operačním systému Debian 11.
Proces vytváření kontejneru webové aplikace využívá třístupňové sestavení. V první fázi dojde ke stažení a instalaci závislostí pomocí nástroje \texttt{npm}.
Tyto závislosti se následně překopírují do druhé fáze, která vytvoří produkční verzi aplikace pomocí nástroje \texttt{next} a vygeneruje potřebné typy pro Prisma ORM.
V poslední fázi se vytvoří kontejner s produkční verzí aplikace, který obsahuje pouze potřebné soubory pro produkční nasazení.
Tento způsob přináší výhodu v rychlosti sestavení, protože se při každé změně
zdrojových kódů nemusí znovu stahovat a instalovat závislosti, ale dochází pouze k sestavení druhé a třetí fáze. Výsledný obraz má velikost přibližně 300 MB.

Pro plánovač byl vytvořen kontejner založený na stejném obrazu. Během druhé fáze dojde k překladu zdrojových souborů v jazyce TypeScript do JavaScriptu pomocí nástroje \texttt{tsc}.
V poslední fázi je vytvořen kontejner s potřebnými soubory bez zdrojových kódů. Sestavený obraz má velikost přibližně 350 MB. 

Databázový kontejner je vytvořen z oficiálního obrazu \texttt{postgres:15}, který také vychází z operačního systému Debian 11 a obsahuje PostgreSQL ve verzi 15.
Ke kontejneru je připojen externí svazek, který obsahuje data databáze. Tento svazek je při prvním spuštění kontejneru vytvořen automaticky a slouží k uchování dat mezi restarty kontejneru.

Posledním kontejner vychází z obrazu \texttt{rabbitmq:3.11} a obsahuje frontu zpráv RabbitMQ, která slouží ke komunikaci mezi webovou aplikací a plánovačem.
Obraz nebylo nutné upravovat.

\subsection{Servisní kontejner}

Pro úvodní nastavení aplikace byl vytvořen kontejner, který
obsahuje skripty potřebné pro inicializaci a správu databáze. Kontejner obsahuje skripty pro vytvoření potřebných tabulek v databázi,
přidání prvního administrátorského účtu, nainstalovaný nástroj Prisma CLI a další.

Jedná se o servisní kontejner určený pro jednorázové interaktivní spuštění. Kontejner umožňuje také přístup k databázi
pomocí nástroje Prisma Studio, který umožňuje správu dat v databázi pomocí webového rozhraní. Pokud by bylo nutné manuálně provést změny v databázi,
je možné spustit kontejner na serveru a pomocí přesměrování příslušného portu na vzdálený server je následně provést potřebné změny z lokálního prohlížeče.
Detailní popis použití kontejneru je uveden v souboru \texttt{README.md} v kořenovém adresáři projektu.