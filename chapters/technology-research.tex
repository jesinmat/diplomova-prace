\chapter{Rešerše technologií}

V této kapitole je popsán výběr vhodných technologií pro implementaci systému. Dostupné technologie jsou uvažovány na základě předchozí analýzy.

\section{Webová aplikace a API}

Webová aplikace bude představovat hlavní rozhraní pro uživatele a společně s API představuje významnou část této práce.
Je proto nutné zaměřit se na výběr vhodné technologie, která umožní splnění všech zadaných požadavků.
Vzhledem k požadavkům na současnou práci více uživatelů při tvorbě plánu není možné využít generování statické webové stránky. Pro klientskou část
aplikace budou proto uvažovány technologie, které umožňují tvorbu dynamických webových aplikací.

Při výběru technologie hrála roli i oblíbenost a žádanost mezi vývojáři. Oblíbený nástroj s širokou komunitou často značí aktivní vývoj, kvalitní dokumentaci
a podporu. Dále je výhodou možnost využití statického typování, které zajišťuje větší bezpečnost a přehlednost kódu. Podle průzkumu Stack Overflow \cite{so_dev_survey} 
byl v minulém roce nejžádanějším frameworkem pro tvorbu webových aplikací React.js. Tento framework je postaven na jazyce JavaScript a umožňuje tvorbu klientských
dynamických webových aplikací. Nabízí také možnost využití statického typování pomocí nástroje TypeScript.

Mezi nejoblíbenějšími technologiemi s více než 1 000 hlasy se umístily také nástroje Svelte, ASP.NET Core a Next.js. Všechny zmíněné technologie umožňují tvorbu dynamických webových aplikací,
klientskou i serverovou část, nabízejí detailní dokumentaci a jsou aktivně vyvíjeny. Podporují také statické typování, částečné generování webových stránek na serveru (\textit{SSR} -- server side rendering),
oddělené serverové částí s podporou přístupu k databázi i tvorbu API.

Pro implementaci byl zvolen framework Next.js ve verzi 13. Tento framework je postaven na React.js a rozšiřuje jej o další funkce, například tvorbu API nebo pokročilejší nastavení serveru.
V době psaní práce podporuje React 18 SSR pouze v beta verzi, Next.js proto nabízí ve výchozím nastavení původní generování webových stránek u klienta.
Během implementace však bude tato funkcionalita využita, protože přináší výhody v rychlosti načítání stránky, možnosti kešování a spouštění kódu na serveru.

Pro přístup k databázi byl zvolen nástroj Prisma. Prisma umožňuje definovat schéma databáze v jednoduché textové podobě a pomocí automatického generování
příslušných typů pro TypeScript zajišťuje bezpečný přístup k databázi. Díky tomu je možné využít statické typování i při práci s databází a není možné vytvořit
neplatný dotaz. Prisma také umožňuje tvorbu migrací, které zajišťují bezpečnou změnu schématu databáze. 

\section{Plánovací systém}

Plánovací systém bude vytvořen v jazyce TypeScript s využitím nástroje Prisma. To umožní minimalizovat počet použitých technologií napříč projektem,
což zjednoduší jeho správu a údržbu. Vzhledem k počtu prací a brigádníků bude možné provést plánování do několika sekund, aby bylo zajištěno splnění požadavku na rychlost.
Plánovač bude vytvořen jako samostatná komponenta, kterou bude možné v případě potřeby upravit přidáním vlastní implementace v jazyce TypeScript
či zcela nahradit jinou implementací. 
Pro přenos požadavků mezi webovým serverem a plánovacím systémem bude využita fronta zpráv.

\section{Fronta zpráv}

Fronta zpráv je datová struktura, která umožňuje asynchronní přenos zpráv mezi jednotlivými komponentami systému. Zprávy jsou ukládány do fronty
a následně zpracovány. V případě chyby je možné zprávu zpracovat znovu. Fronta zpráv je vhodným řešením pro přenos zpráv mezi webovým serverem a plánovacím systémem,
protože zajišťuje spolehlivý přenos zpráv v případě, že jedna ze služeb není v danou chvíli krátkodobě dostupná.

Službou pro poskytování fronty zpráv byla zvolena technologie RabbitMQ. Jedná se o open-source nástroj, který je dostupný zdarma. RabbitMQ je běžně používaným řešením
pro fronty zpráv. Využívá otevřený standard AMQP, jehož implementace je pro mnoho jazyků, včetně TypeScriptu, dostupná zdarma. TODO: zdroj "https://www.rabbitmq.com/tutorials/tutorial-one-javascript.html".


\section{Databáze}

Pro výběr databáze byla na základě struktury dat zvolena technologie relačních databází. Uložená data mají pevně danou neměnnou strukturu,
jedná se o krátké textové řetězce, čísla a datum. Výhodou relačních databází je možnost snadno definovat vztahy mezi jednotlivými tabulkami,
což je v případě plánovacího systému nezbytné. Dále je výhodou možnost využití transakcí, které zajišťují konzistenci dat.
Na základě analýzy a interních zkušeností administrátorského týmu akce SummerJob byl zvolen databázový systém PostgreSQL. Tento systém je
dostupný zdarma, má otevřený zdrojový kód a jedná se o běžně používané řešení s historií dlouhou více než 35 let. TODO: zdroj "https://www.postgresql.org/about/".
Podle uživatelského průzkumu Stack Overflow se v minulém roce jednalo o nejoblíbenější a nejžádanější databázový systém. TODO: zdroj  https://survey.stackoverflow.co/2022/\#section-most-loved-dreaded-and-wanted-databases

Objem dat a počet současných uživatelů systému je relativně malý, proto není nutné využívat technologie pro škálování databáze. Zálohování a obnova dat
je možná přes externí nástroje, které jsou také dostupné zdarma. Jedná se například o aplikaci pgAdmin, která umožňuje správu databáze přes grafické rozhraní.
Prisma poskytuje podporu pro PostgreSQL verze 9.4 a vyšší.

\section{Kontejnerizace}

Pro zajištění přenositelnosti a nezávislosti na platformě podle požadavku NF3 budou všechny komponenty systému spouštěny v kontejnerech. TODO: zdroj "https://www.ibm.com/topics/containers".
Kontejnerizace je technologie, která umožňuje spouštět aplikace v izolovaném prostředí. Aplikace spuštěná v kontejneru je izolována od ostatních aplikací,
které jsou spuštěny na hostitelském operačním systému nebo v jiných kontejnerech. Kontejnery obsahují
všechny potřebné závislosti, které jsou nutné pro spuštění aplikace. Díky tomu je možné aplikaci spustit na libovolném operačním systému, který podporuje kontejnerizaci.

Pro kontejnerizaci byla zvolena technologie Docker. Jedná se o open-source nástroj, který je dostupný zdarma pro všechny běžné operační systémy.
V roce 2022 se jednalo o nejpoužívanější technologii pro kontejnerizaci. TODO: zdroj "https://www.statista.com/statistics/1256245/containerization-technologies-software-market-share/".

