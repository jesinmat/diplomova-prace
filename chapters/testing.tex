\chapter{Testování}

Tato kapitola popisuje proces testování aplikace. Aplikace byla průběžně testována uživateli během vývoje, API bylo testováno pomocí automatizovaných testů.

\section{Testování uživateli}

Od začátku vývoje byla aplikace dostupná na dočasné webové adrese, ke které měli přístup organizátoři akce SummerJob. Tito organizátoři aplikaci průběžně testovali,
čímž ověřovali splnění funkčních požadavků a poskytovali zpětnou vazbu. Na testování se podíleli zejména vedoucí práce, Ing. Marek Jílek, a Bc. Lucie Anna Procházková,
dlouholetí organizátoři akce SummerJob. Dále se na zkušebním provozu mohli podílet i další organizátoři akce SummerJob, kteří aplikaci testovali v menší míře.

Pomocí tohoto procesu bylo zjištěno několik nedostatků, které byly následně odstraněny. Jednalo se zejména o specifické požadavky pro akci SummerJob, které nebyly
v původním návrhu specifikovány. 

Pro testovací účely byla pro aplikaci generována data pomocí nástroje Faker.js \cite{fakerjs}. Tento nástroj umožňuje generovat náhodná data v různých formátech, například jména,
emailové adresy, adresy, telefonní čísla a další. Díky tomu bylo možné vytvořit testovací data, která byla podobná reálným datům, a tím lépe otestovat
funkčnost aplikace. Ukázka generování náhodných dat je uvedena v \ref{code:faker}.

\begin{listing}[h]
  \begin{minted}{javascript}
import { faker } from "@faker-js/faker/locale/cz";

await prisma.worker.create({
  data: {
    firstName: faker.name.firstName(),
    lastName: faker.name.lastName(),
    phone: faker.phone.number("### ### ###"),
    email: faker.internet.email(firstName, lastName).toLocaleLowerCase(),
    isStrong: Math.random() > 0.75
  }
})
    \end{minted}
    \caption{Kontrola oprávnění pro přístup ke stránkám ve skupině Auta}
    \label{code:faker}
\end{listing}

\section{Testování API}

Pro zajištění správné funkcionality API byly vytvořeny automatizované testy. Tyto testy ověřují, že API vrací očekávaná data a že nedochází k chybám při zpracování požadavků.
Testy byly vytvořeny pomocí knihoven \texttt{Chai} \cite{chai} a \texttt{Mocha} \cite{mocha}, rozšířené o knihovnu \texttt{supertest} \cite{supertest} pro testování HTTP požadavků.
Mocha je rozšířeným nástrojem pro testování JavaScriptových aplikací, který umožňuje vytvářet testovací sady a testy. Podporuje asynchronní testování a umožňuje
kombinovat různé styly zápisu testů. Chai nabízí více možností zápisu validace dat, v testech je využit styl BDD\footnote{Behavior Driven Development}. Tento styl umožňuje používat přirozený jazyk
pro popis testů, což zvyšuje čitelnost testů. Supertest je knihovna, která umožňuje testovat HTTP požadavky. V testech je využívána pro odesílání požadavků na API.

Vzhledem k použité metodě přihlašování není možné automaticky generovat přístupový token pomocí knihovny NextAuth.js, na začátku testů je tedy v databázi automaticky
vytvořen uživatel s administrátorským účtem a do tabulky aktivních přihlášení je testovacím skriptem vložen uměle vytvořený token. 

Testování probíhá na produkční verzi webové aplikace s prázdnou databází. Testy pokrývají téměř všechny dostupné endpointy API, aby bylo možné
ověřit, že všechny funkce API fungují správně. Výjimku tvoří endpointy pro přihlášení pomocí e-mailu, které vzhledem k povaze přihlášení nelze 
pomocí API testů snadno otestovat. Pro kontrolu jednotlivých API endpointů je možné využít nástroj Swagger UI, který je dostupný ve vývojovém režimu aplikace.

\subsection{Ukázka testů}

Test v ukázce \ref{code:api-test} ověřuje funkčnost požadavku na odstranění plánu.
Nejprve je vytvořen plán, následně je získán seznam všech plánů a po odstranění plánu je získán seznam plánů znovu a je ověřeno, že plán byl odstraněn.
Testy využívají asynchronní zápis pomocí klíčového slova \texttt{async} a \texttt{await}, které umožňuje zapisovat asynchronní kód jako synchronní bez zanořování.


\begin{listing}[h]
\begin{minted}{javascript}
describe("Plans", function () {
  it("deletes a plan", async function () {
    // Create a plan
    const plan = await api.post(
      "/api/plans",
      Id.PLANS,
      createPlanData(api.getSummerJobEventStart())
    );
    // Get all plans before the plan is deleted
    const plansBefore = await api.get("/api/plans", Id.PLANS);
    // Delete the created plan
    const deleted = await api.del(
        `/api/plans/${plan.body.id}`,
        Id.PLANS
    );
    deleted.status.should.equal(204);
    // Check that the plan was deleted from the list
    const plansAfter = await api.get("/api/plans", Id.PLANS);
    plansAfter.body
        .should.have.lengthOf(plansBefore.body.length - 1);
    plansAfter.body.map(_plan => _plan.id)
        .should.not.contain(plan.body.id);
  });
});
\end{minted}
\caption{Test API pro odstranění plánu}
\label{code:api-test}
\end{listing}

\section{Ověření splnění požadavků}

V této sekci jsou ověřeny požadavky na aplikaci, které byly definovány v sekcích \ref{sec:functional-requirements} a \ref{sec:nonfunctional-requirements}.


\subsection{Funkční požadavky}

\begin{itemize}
  \item \textbf{FP1. Přihlášení a registrace:} přihlášení bylo implementováno pomocí odkazu zaslaného na e-mailovou adresu účastníka. Registrace v souladu se zadáním není součástí aplikace, organizátoři importují účty účastníku pomocí webového rozhraní aplikace.
  \item \textbf{FP2. Správa oprávnění a uživatelských účtů:} sekce systému a API byly rozděleny podle potřebného oprávnění. Tato oprávnění může spravovat pouze uživatel s 
  administrátorským oprávněním.
  \item \textbf{FP3. Správa pracantů:} požadované údaje jsou evidovány a webová aplikace umožňuje jejich úpravu. Je k dispozici individuální přidání pracanta i hromadný import.
  \item \textbf{FP4. Správa jobů:} joby je možné spravovat pomocí webové aplikace podle zadaných požadavků.
  \item \textbf{FP5. Správa aut:} požadované údaje jsou ve webové aplikaci zobrazeny a je možné s nimi pracovat.
  \item \textbf{FP6. Správa plánů:} implementace podporuje práci více uživatelů pomocí aktualizace dat na pozadí a zobrazení změn s minimálním zpožděním. 
  \item \textbf{FP7. Automatický plánovač:} automatický plánovač byl implementován.
  \item \textbf{FP8. Správa ročníků:} správa ročníků byla implementována jako součást administrační části webové aplikace.
  \item \textbf{FP9. Správa oblastí:} aplikace umožňuje spravovat oblasti ročníku.
  \item \textbf{FP10. Záznam systémových změn:} požadavky, které mění data v databázi, jsou zaneseny do historie požadavků a evidují se požadované údaje. Požadavky
  lze zobrazit v administraci.
  \item \textbf{FP11. API:} systém poskytuje API, které umožňuje přístup k datům systému. Webová aplikace využívá API k provádění operací, jedná se tedy o plnohodnotné API.
\end{itemize}

\subsection{Nefunkční požadavky}

\begin{itemize}
  \item \textbf{NP1. Bezpečnost:} webová aplikace i API vyžadují přihlášení. Databáze je spuštěna v kontejneru, který je připojen do společné sítě s webovou aplikací,
  ale nemá přímý přístup k internetu. Aplikace bude uvedena do provozu na doméně akce SummerJob, která využívá HTTPS. V případě pokusu o přístup přes nešifrované spojení
  musí dojít k přesměrování na šifrované spojení. Další parametry nasazení do produkčního prostředí jsou popsány v následující kapitole.
  \item \textbf{NP2. Udržovatelnost a rozšiřitelnost:} výsledná aplikace využívá rozšířené technologie a standardy, například React a Bootstrap ve webové aplikaci,
  PostgreSQL pro ukládání dat a Docker pro kontejnerizaci. Všechny tyto technologie jsou dobře zdokumentované a aktivně vyvíjené. Zdrojový kód je rozčleněn do logických celků
  a API nabízí dokumentaci ve formátu Swagger UI a OpenAPI standardu.
  \item \textbf{NP3. Nezávislost na platformě:} aplikace je díky kontejnerizace spustitelná na různých operačních systémech. Využití knihovny Bootstrap zajišťuje,
  že webová aplikace je použitelná na mobilních zařízeních i počítačích.
  \item \textbf{NP4. Rychlost plánovače:} během opakovaného používání plánovače bylo změřeno, že plánovač je schopen vygenerovat plán pro 150 pracantů a 50 jobů přibližně do 6 sekund.
  Toto měření zahrnuje čas od zadání požadavku přes webové rozhraní až do zobrazení výsledného plánu a obsahuje tedy i komunikaci mezi klientem a serverem. Vzhledem k výrazně 
  nižšímu času oproti požadavku nebylo měření času provedeno exaktně pomocí automatizovaných testů na plánovači, neboť z výsledku vyplývá, že plánovač je dostatečně rychlý.
\end{itemize}

Všechny požadavky byly splněny. Výsledná aplikace je funkční a splňuje požadavky zadání. Dodatečné bezpečnostní prvky týkající se komunikace mezi uživatelem
a webovou aplikací je možné nastavit pomocí konfigurace webového serveru, na kterém bude aplikace provozována.
