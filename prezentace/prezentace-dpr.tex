% arara: xelatex : {options: ["-shell-escape"]}
% dokumentace k beameru: http://ftp.cvut.cz/tex-archive/macros/latex/contrib/beamer/doc/beameruserguide.pdf

% nastavení formátu prezentace 16:9 
\documentclass[czech,aspectratio=169]{beamer}

\usepackage{polyglossia}
\setmainlanguage{czech}

% nastavení vzhledu 
% další možnosti vzhledu viz https://hartwork.org/beamer-theme-matrix/
\usetheme{Madrid}
\usecolortheme{whale}

% vzhled slajdů vnitřní téma (např. vzhled odrážek)
\useinnertheme{rectangles} %možnosti: default circles rectangles rounded inmargin
% vzhled slajdů vnější téma
\useoutertheme{default} %možnosti: default, miniframes, smoothbars, sidebar, split, shadow, tree, smoothtree, infolines

% zavedeme čvutí modou barvu
\definecolor{CVUT}{HTML}{0065BD}
% čvutí modou použijeme jako hlavní barvu prezentace
\setbeamercolor{structure}{bg=white,fg=CVUT}

% jako font prezentace nadefinujeme oficiální ČVUT písmo Technika -- pokud chcete použít, musíte si font nainstalovat nebo jej nahrát na Overleaf
% https://www.cvut.cz/logo-a-graficky-manual  -- inforek, přihlášení přes celoškolské heslo
%\usepackage{fontspec}
%\setsansfont{Technika-Kniha}

% vypneme navigační panel beamer (pro zapnutí zakomentujeme)
\beamertemplatenavigationsymbolsempty

% vygenerujeme slajdy s poznámkami -- ty si můžete vytisknout a mít je na obhajobu s sebou (pokud zapomenete slova, nebo kdyby nefungovalo promítání z nějakého důvodu)
%\setbeameroption{show notes} 

% vygeneruje slajdy s poznámky vhodné pro promítání na dvou monitorech -- na obhajobu nevyužijete
%\usepackage{pgfpages}
%\setbeameroption{show notes on second screen}

% další balíčky
\usepackage{graphicx}
\usepackage{minted}
\usepackage{hyperref}
\usepackage{tikz}
\usepackage{pgfplots}
\usetikzlibrary{chains,fit,shapes}
\pgfplotsset{compat=1.15}


% Údaje o prezentaci
\title[Aplikace pro SummerJob]{Návrh a implementace aplikace pro dobrovolnickou brigádu SummerJob}
\subtitle{Diplomová práce}
\institute[FIT ČVUT v Praze]{Fakulta informačních technologií \\ České vysoké učení technické v Praze}
\author[M. Ješina]{Bc. Matyáš Ješina \\ Vedoucí práce: Ing. Marek Jílek}
\date{15. 6. 2023}
\titlegraphic{\includegraphics[width=.1\textwidth]{logo-cvut}}


\begin{document}


  \begin{frame}
    \titlepage 
    \note{Nezapomenout pozdravit} %tohle je poznámka, ta na slajdu nebude, ale vygeneruje se vedle něj, pokud odkomentujete příkaz výše -- \setbeameroption{show notes} 
  \end{frame}
  
  % \begin{frame}
  %   \tableofcontents %generuje se automaticky z section, subsection, subsubsection
  % \end{frame}

  \section{Co je SummerJob?}
  \begin{frame}{Co je SummerJob?}
    \begin{columns}
      \begin{column}{.65\textwidth}
        \begin{itemize}
          \item Letní dobrovolnická brigáda v ČR.
          \item Přibližně 150 dobrovolníků ročně.
          \item Různé typy prací.
          \item Pomoc těm, kteří na práci sami nestačí.
        \end{itemize}
      \end{column}
  \begin{column}{.3\textwidth}
    \includegraphics[width=0.8\textwidth]{summerjob-logo}
  \end{column}
\end{columns}
  \end{frame}
  
  \section{Cíl práce}
  \begin{frame}{Cíl práce}
    Vytvořit informační systém pro organizaci. Musí obsahovat:
    \begin{itemize}
      \item evidenci prací a účastníků,
      \item plánování prací včetně dopravy,
      \item rozhraní pro organizátory i účastníky,
      \item možnost spolupráce více organizátorů současně.
    \end{itemize}
    Systém musí být dostupný i na mobilních zařízeních.
  \end{frame}

  \begin{frame}{Stávající řešení}
    Pro plánování byla v minulosti používána samostatná aplikace. Nevýhody:
    \begin{itemize}
      \item dostupná pouze na PC,
      \item jeden administrátorský účet,
      \item neumožňovala současnou práci více lidí,
      \item pomalé automatické plánování (50 s),
      \item nebyla dostupná pro účastníky.
    \end{itemize}
  \end{frame}

  \section{Vývoj}
  \begin{frame}{Výběr operačního systému}
    \begin{columns}
      \begin{column}{.6\textwidth}
        Pro tvorbu obrazu jsem zvolil TurnKey GNU/Linux, založený na distribuci Debian.
        \begin{itemize}
          \item<2-> jednoduchý na použití,
          \item<3-> automatizovatelný,
          \item<4-> bezpečný,
          \item<5-> rychlý,
          \item<6-> malý.
        \end{itemize}
      \end{column}
      \begin{column}{.3\textwidth}
        \begin{itemize}
          \item[]<1->{\includegraphics[width=0.8\textwidth]{turnkey}}
        \end{itemize}
      \end{column}
    \end{columns}
  \end{frame}

  \begin{frame}{Správce herních serverů}
    Program, který se stará o instalaci a provoz herního serveru.
  \end{frame}

  \begin{frame}{Správce herních serverů}{LinuxGSM}
    \begin{columns}
      \begin{column}{.6\textwidth}
        Sada skriptů LinuxGSM (Linux Game Server Managers) podporuje přes 100 herních serverů.
        \begin{itemize}
          \item stažení,
          \item instalace,
          \item spuštění,
          \item monitorování,
          \item zastavení.
        \end{itemize}
      \end{column}
      \begin{column}{.3\textwidth}
          \includegraphics[width=0.7\textwidth]{linuxgsm}
      \end{column}
    \end{columns}
  \end{frame}

  \begin{frame}{Interaktivní výběr serveru}
    \centering
    \includegraphics[width=0.7\paperwidth]{gameservers}
  \end{frame}

  \begin{frame}{Automatizace}
    Je nutné automatizovat tyto kroky:
    \begin{itemize}
      \item sestavení obrazu systému, \pause
      \item připravení obrazu pro provoz v cloudu, \pause
      \item instalace herního serveru, \pause
      \item spuštění herního serveru.
    \end{itemize}
  \end{frame}

  \begin{frame}{Zveřejnění výsledků}
    \begin{itemize}
      \item TurnKey Hub - obrazy
      \item GitHub - zdrojové kódy
    \end{itemize}
  \end{frame}

  \begin{frame}{Možnosti rozšíření}
    Všechny zdrojové kódy jsou volně dostupné, je možné:
    \begin{itemize}
      \item provozovat systém u sebe nebo v cloudu,
      \item jednoduše přidávat podporu pro nové herní servery,
      \item provozovat herní servery v komerčním prostředí.
    \end{itemize}
  \end{frame}

  \section{Shrnutí}
  \begin{frame}{Shrnutí}
    \begin{itemize}
      \item Cílem práce bylo vytvořit obraz systému pro herní servery v cloudu. 
      \item Obraz lze sestavit s využitím existujících nástrojů.
      \item Podpora až 100 různých herních serverů s možností rozšíření.
      \item Systém splňuje bezpečnostní požadavky.
      \item Celý proces je plně automatizovatelný.
    \end{itemize}
  \end{frame}

\end{document}
